TODO: Nuno, please summarize these recent papers.  You can do this
only superficially on a first pass.  But it is important to do it
quickly to avoid surprises.  Some questions: Are they dealing with the
same problem as we are?  If no, is this problem really relevant?  If
yes, what is the advantage of our approach compared to theirs?

%%%%%%%
Which Configuration Option Should I Change?  ICSE'14

This paper tries to identify the configuration options that need to be changed/modified
so that a new version of the software behaves as an old version. 

Our solution does not need an old version in order to do the analysis. 

%%%%%%%
Automated Diagnosis of Software Configuration Errors.  ICSE'13

We are trying to address the same problem, but in a different way. They solve it by using
static analysis (thin slices) combined with dynamic analysis (of predicates). Instead, we
use a logic reasoning approach (we can identify multiple configuration errors, for instance).

As a major disadvantage of their approach is the fact that they need a rather complete pool
of correct executions in order for the approach to be able to identify what is erroneous.


%%%%%%%
Do Not Blame Users for Misconfigurations.  SOSP'13

%%%%%%%
Failure Avoidance in Configurable Systems through Feature Locality. Assurances for Self-Adaptive Systems 2013
